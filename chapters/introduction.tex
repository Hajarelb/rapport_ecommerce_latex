\chapter*{}
\addcontentsline{toc}{chapter}{Introduction générale}
\markboth{Introduction générale}{}

\begin{center}
	{\Huge\bfseries Introduction générale}
\end{center}

\vspace{0.5cm}

L’évolution rapide des technologies numériques a profondément transformé les modes de consommation et les pratiques commerciales. Le commerce électronique, communément appelé e-commerce, s’est imposé comme un élément central de l’économie moderne. Il permet aux utilisateurs d’accéder à des services et à des produits à distance, tout en offrant aux entreprises de nouvelles opportunités de gestion, de visibilité et de croissance. Cette transformation repose en grande partie sur le développement d’applications informatiques distribuées capables de garantir performance, fiabilité et sécurité.

\vspace{0.5cm}

Les applications distribuées jouent aujourd’hui un rôle fondamental dans la conception des systèmes modernes. Elles permettent la séparation des responsabilités, la gestion efficace des données et la communication entre plusieurs composants logiciels. Dans le domaine du e-commerce, ces caractéristiques sont essentielles pour assurer la disponibilité des services, la protection des informations sensibles et une expérience utilisateur fluide.

\vspace{0.5cm}

Dans le cadre du module « Applications Distribuées », nous avons été amenés à concevoir et développer une mini-application e-commerce illustrant les principes étudiés durant le semestre. Ce projet vise à mettre en pratique les concepts théoriques liés aux architectures client-serveur, à la structuration d’une application web moderne, ainsi qu’à la sécurisation des échanges entre les différents composants du système.

\vspace{0.5cm}

L’application développée propose un ensemble de fonctionnalités essentielles à une plateforme de vente en ligne : gestion des produits, authentification des utilisateurs, manipulation du panier d’achat et traitement des commandes. Elle repose sur un backend construit avec Spring Boot et Spring MVC, une interface web utilisant Thymeleaf, ainsi qu’une base de données relationnelle MySQL. La sécurité est assurée par l’utilisation de JSON Web Tokens (JWT), garantissant un accès contrôlé aux ressources.

\vspace{0.8cm}

Dans le premier chapitre, nous présentons le cadre général du projet ainsi que les notions fondamentales liées aux applications distribuées et au commerce électronique. Nous y introduisons les concepts théoriques nécessaires à la compréhension de l’architecture adoptée, notamment le modèle client-serveur, la séparation des couches applicatives et les principes de communication entre les composants d’un système distribué.

\vspace{0.5cm}

Le deuxième chapitre est consacré à l’analyse et à la conception du système. Il décrit les besoins fonctionnels et non fonctionnels de l’application, les différents acteurs intervenant dans le système, ainsi que la modélisation UML comprenant les diagrammes de cas d’utilisation et le diagramme de classes. Cette phase permet de structurer l’application avant son implémentation et d’assurer une vision claire de son organisation interne.

\vspace{0.5cm}

Le troisième chapitre détaille l’architecture technique et l’environnement de développement. Il présente les technologies utilisées, notamment Spring Boot, Spring MVC, Thymeleaf et MySQL, ainsi que le mécanisme d’authentification basé sur JWT. Ce chapitre met en évidence la structuration en couches (contrôleurs, services, accès aux données) et explique le rôle de chaque composant dans le fonctionnement global de l’application.

\vspace{0.5cm}

Enfin, le quatrième chapitre expose la phase d’implémentation et les principaux résultats obtenus. Il décrit les fonctionnalités réalisées, les interfaces développées et les interactions entre le frontend et le backend. Les difficultés rencontrées ainsi que les solutions adoptées y sont également présentées.

\vspace{0.5cm}

Ce rapport se termine par une conclusion générale qui synthétise les apports pédagogiques du projet et met en lumière les perspectives d’amélioration envisageables.

\newpage
