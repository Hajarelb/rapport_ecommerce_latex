\cleardoublepage
\thispagestyle{empty}
\markboth{Chapitre 4 : Implémentation et réalisation}{}

\refstepcounter{chapter}
\addcontentsline{toc}{chapter}{Implémentation et réalisation}


\vspace*{3cm}

\begin{center}
	{\Huge\bfseries CHAPITRE \thechapter}
	
	\vspace{0.8cm}
	\rule{10cm}{1.2pt}
	
	\vspace{0.8cm}
	{\LARGE Implémentation et réalisation}
	
	\vspace{0.8cm}
	\rule{10cm}{1.2pt}
\end{center}

\vspace*{\fill}

\newpage


\section{Introduction}

Ce chapitre présente la concrétisation technique de l’application e-commerce. Il décrit les technologies utilisées, l’architecture mise en place et les interfaces principales développées.

\section{Architecture de l’application}

L’application adopte une architecture modulaire séparant le frontend et le backend :

\begin{itemize}
	\item Backend : gestion de la logique métier, base de données et API
	\item Frontend : interface utilisateur interactive
\end{itemize}

Cette séparation améliore la maintenabilité et l’évolutivité du système.

\section{Technologies utilisées}

\subsection{Backend}

\begin{itemize}
	\item Framework serveur pour la gestion des services
	\item API REST pour la communication
	\item Base de données relationnelle pour le stockage
\end{itemize}

\subsection{Frontend}

\begin{itemize}
	\item Bibliothèque JavaScript moderne pour l’interface
	\item HTML/CSS pour la présentation visuelle
\end{itemize}

\subsection{Sécurité}

\begin{itemize}
	\item Authentification utilisateur
	\item Protection des données
\end{itemize}

\section{Fonctionnalités implémentées}

\begin{itemize}
	\item Gestion des comptes utilisateurs
	\item Gestion des produits
	\item Gestion du stock
	\item Création de commandes
	\item Suivi des transactions
\end{itemize}

\section{Interfaces principales}

Les interfaces développées privilégient :

\begin{itemize}
	\item Simplicité d’utilisation
	\item Navigation intuitive
	\item Rapidité d’accès aux fonctionnalités
\end{itemize}

Des tableaux de bord facilitent la gestion quotidienne.

\section{Tests et validation}

Des tests fonctionnels ont été réalisés pour garantir :

\begin{itemize}
	\item Cohérence des données
	\item Stabilité du système
	\item Sécurité des accès
\end{itemize}

\section{Conclusion}

L’implémentation confirme la faisabilité technique du projet. Le système répond aux objectifs définis tout en restant extensible.
