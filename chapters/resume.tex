\chapter*{}
\addcontentsline{toc}{chapter}{Résumé}

\begin{center}
	{\Huge\bfseries Résumé}
\end{center}

\vspace{0.5cm}

Ce rapport présente le travail réalisé dans le cadre du mini-projet du module « Applications Distribuées » 
au sein de la Faculté Polydisciplinaire à Larache.

\vspace{0.5cm}

Dans un contexte où le commerce électronique connaît une croissance rapide, 
les plateformes en ligne permettent aux utilisateurs d’acheter des produits 
de manière simple, rapide et sécurisée. 
Ces applications reposent sur des architectures distribuées 
assurant la gestion efficace des données, la communication entre les différentes couches du système 
et la sécurisation des transactions.

\vspace{0.5cm}

L’objectif principal de ce projet est de concevoir et développer une application e-commerce 
permettant la gestion des produits, des utilisateurs et des commandes, 
tout en mettant en œuvre une architecture backend structurée et sécurisée.

\vspace{0.5cm}

Pour ce faire, nous avons utilisé Spring Boot et Spring MVC pour le développement du backend, 
Thymeleaf pour la couche de présentation, ainsi qu’une base de données MySQL pour la persistance des données. 
La sécurisation de l’application a été assurée par l’utilisation de JSON Web Token (JWT).

\vspace{0.8cm}

\textbf{Mots clés :} application web, e-commerce, applications distribuées, Spring Boot, Thymeleaf, MySQL, JWT.
