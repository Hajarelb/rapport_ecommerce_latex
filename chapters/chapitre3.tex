\cleardoublepage
\thispagestyle{empty}
\markboth{Chapitre 3 : Spécification et modélisation}{}

\refstepcounter{chapter}
\addcontentsline{toc}{chapter}{Spécification et modélisation}


\vspace*{3cm}

\begin{center}
	{\Huge\bfseries CHAPITRE \thechapter}
	
	\vspace{0.8cm}
	\rule{10cm}{1.2pt}
	
	\vspace{0.8cm}
	{\LARGE Spécification et modélisation}
	
	\vspace{0.8cm}
	\rule{10cm}{1.2pt}
\end{center}

\vspace*{\fill}

\newpage

\section{Introduction}

Après l’analyse des besoins et la définition du cadre du projet, cette étape vise à formaliser les spécifications fonctionnelles et techniques de l’application. La modélisation permet de traduire les exigences en représentations structurées afin de faciliter la conception et le développement.

Dans le cadre de notre application e-commerce de gestion commerciale, cette modélisation décrit les acteurs impliqués, les interactions avec le système, ainsi que l’organisation des données.

\section{Identification des acteurs}

L’application repose sur plusieurs profils utilisateurs disposant de droits spécifiques :

\begin{itemize}
	\item \textbf{Administrateur} : gère les utilisateurs, les produits, les commandes et supervise l’ensemble du système.
	\item \textbf{Gestionnaire} : responsable des ventes, du stock et des commandes.
	\item \textbf{Client} : consulte les produits, passe des commandes et suit ses achats.
\end{itemize}

Cette séparation garantit une gestion sécurisée et adaptée aux responsabilités de chaque utilisateur.

\section{Diagrammes de cas d’utilisation}

Les diagrammes de cas d’utilisation illustrent les interactions entre les acteurs et le système :

\begin{itemize}
	\item Authentification et gestion des comptes
	\item Gestion des produits
	\item Gestion du stock
	\item Gestion des commandes
	\item Consultation de l’historique
\end{itemize}

Ces diagrammes permettent de visualiser les fonctionnalités accessibles selon chaque profil utilisateur.

\section{Modélisation des données}

Le diagramme de classes définit les principales entités :

\begin{itemize}
	\item Utilisateur
	\item Produit
	\item Commande
	\item Client
	\item Paiement
\end{itemize}

Chaque entité contient des attributs spécifiques permettant une gestion cohérente des informations commerciales.

\section{Diagrammes de séquence}

Les diagrammes de séquence décrivent le déroulement des interactions pour des scénarios clés :

\begin{itemize}
	\item Connexion utilisateur
	\item Passage de commande
	\item Mise à jour du stock
\end{itemize}

Ils illustrent la communication entre les composants du système.

\section{Conclusion}

La modélisation fournit une vision claire de l’architecture fonctionnelle du système. Elle constitue une base solide pour l’implémentation en assurant cohérence, organisation et évolutivité.
