\chapter*{}
\addcontentsline{toc}{chapter}{Conclusion générale}
\markboth{Conclusion générale}{} % Met à jour les en-têtes pour ce chapitre

\begin{center}
	{\Huge\bfseries Conclusion générale}
\end{center}

\vspace{0.5cm}

Notre projet de fin de module, réalisé au sein du département d’Informatique de la Faculté Polydisciplinaire de Larache, a porté sur la conception et le développement d’une application e-commerce distribuée intégrant des mécanismes de sécurité avancés et un déploiement en environnement Cloud.

Cette expérience a constitué une opportunité concrète de mettre en pratique les connaissances théoriques acquises durant notre formation, notamment dans les domaines des applications distribuées, de l’architecture logicielle et de la sécurité informatique.

À travers les différentes phases du projet — analyse des besoins, modélisation UML, conception architecturale, développement, sécurisation et déploiement — nous avons pu nous confronter aux exigences réelles de la conception d’un système web complet. L’intégration de technologies telles que Spring Boot, Spring Security avec JWT et MySQL nous a permis d’approfondir nos compétences techniques et de mieux comprendre les enjeux liés à la performance, à la sécurité et à la scalabilité d’une application moderne.

Au-delà de l’aspect technique, ce projet nous a également permis de renforcer nos compétences en gestion du temps, en organisation méthodique du travail et en collaboration au sein d’une équipe. La planification à travers un diagramme de Gantt et la répartition claire des responsabilités ont contribué à assurer une progression cohérente et structurée.

Bien que l’application développée réponde aux objectifs fixés, plusieurs perspectives d’amélioration peuvent être envisagées, telles que l’intégration d’un système de paiement en ligne, la migration vers une architecture microservices, l’automatisation complète des tests ou encore la containerisation via Docker. Ces évolutions permettraient d’enrichir davantage la solution et de la rapprocher des standards industriels actuels.

Nous exprimons enfin notre gratitude envers la Faculté Polydisciplinaire de Larache et l’équipe pédagogique pour l’encadrement et les connaissances transmises tout au long de ce module. Cette expérience représente une étape importante dans notre parcours académique et constitue une base solide pour notre future insertion professionnelle dans le domaine des systèmes distribués et du développement web avancé.