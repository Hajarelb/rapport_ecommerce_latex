\cleardoublepage
\thispagestyle{empty}
\markboth{Chapitre 2 : Déroulement du projet}{} % Met à jour les en-têtes pour ce chapitre

\refstepcounter{chapter}
\addcontentsline{toc}{chapter}{Déroulement du projet}


\vspace*{3cm}

\begin{center}
	{\Huge\bfseries CHAPITRE \thechapter}
	
	\vspace{0.8cm}
	\rule{10cm}{1.2pt}
	
	\vspace{0.8cm}
	{\LARGE Déroulement du projet}
	
	\vspace{0.8cm}
	\rule{10cm}{1.2pt}
\end{center}

\vspace*{\fill}

\newpage

\section{Introduction}

Ce chapitre présente le déroulement du projet depuis sa planification initiale jusqu’à sa mise en œuvre finale. Il décrit l’organisation des tâches, la méthodologie adoptée et les différentes phases du développement. L’objectif est de montrer comment le projet a été structuré afin d’assurer une progression logique, maîtrisée et conforme aux objectifs pédagogiques du module « Applications Distribuées ».

\section{Planification opérationnelle}

Afin d’assurer une gestion rigoureuse du projet, une planification détaillée a été réalisée sous forme de diagramme de Gantt.  

Ce diagramme permet de visualiser :
\begin{itemize}
    \item Les différentes phases du projet
    \item La durée de chaque tâche
    \item L’enchaînement chronologique des activités
    \item La répartition des responsabilités entre les membres de l’équipe
\end{itemize}

Le projet s’est déroulé sur une période de trois semaines (Janvier – Février 2026), selon une organisation progressive et structurée.

\subsection{Diagramme de Gantt}

\begin{figure}[H]
    \centering
    \includegraphics[width=1\textwidth]{figures/gantt.png}
    \caption{Diagramme de Gantt du projet E-Commerce}
    \label{fig:gantt}
\end{figure}

\section{Phases du développement}

Le projet a été structuré en six phases principales, conformément au diagramme de Gantt présenté précédemment.

\subsection*{Phase 1 : Analyse et Spécification}

Cette phase initiale a été consacrée à :
\begin{itemize}
    \item L’analyse des besoins fonctionnels
    \item L’étude de l’existant
    \item La rédaction des spécifications
\end{itemize}

Elle a permis d’établir une base claire pour la conception du système.

\subsection*{Phase 2 : Conception et Modélisation}

Durant cette phase, les éléments suivants ont été réalisés :
\begin{itemize}
    \item Diagrammes de cas d’utilisation
    \item Diagramme de classes
    \item Diagrammes de séquence
    \item Définition de l’architecture applicative
\end{itemize}

Cette étape a permis de structurer le système avant son implémentation.

\subsection*{Phase 3 : Mise en place de l’environnement}

Cette phase a consisté à :
\begin{itemize}
    \item Configurer Spring Boot et MySQL
    \item Mettre en place la configuration de sécurité JWT
    \item Préparer l’environnement de développement
\end{itemize}

Elle a assuré une base technique stable pour le développement.

\subsection*{Phase 4 : Développement}

Le développement a été organisé en modules :

\begin{itemize}
    \item Module d’authentification (JWT)
    \item Module de gestion des produits
    \item Module panier et commandes
    \item Module administration et tableau de bord
    \item Gestion des codes promotionnels
    \item Interface frontend avec Thymeleaf
\end{itemize}

Chaque module a été développé et testé progressivement.

\subsection*{Phase 5 : Tests et Validation}

Cette phase a inclus :
\begin{itemize}
    \item Tests fonctionnels
    \item Correction des anomalies détectées
\end{itemize}

Elle a permis d’assurer la stabilité et la fiabilité de l’application.

\subsection*{Phase 6 : Documentation et Rapport}

La dernière phase a été consacrée à :
\begin{itemize}
    \item La rédaction du rapport final
    \item La mise en forme et validation du document
\end{itemize}

Cette étape clôture officiellement le projet.

\section{Conclusion}

Ce chapitre a présenté le déroulement méthodologique du projet à travers une planification structurée et une organisation progressive des tâches.

Le diagramme de Gantt met en évidence une gestion rigoureuse du temps et une répartition claire des responsabilités.  

L’approche incrémentale adoptée a permis de développer, tester et intégrer les différentes fonctionnalités de manière maîtrisée, tout en garantissant la cohérence architecturale du système.

Cette organisation méthodique constitue une base solide pour la phase de spécification et de modélisation détaillée dans le chapitre suivant.
