\cleardoublepage
\thispagestyle{empty}
\markboth{Chapitre 2 : Déroulement du projet}{} % Met à jour les en-têtes pour ce chapitre

\refstepcounter{chapter}
\addcontentsline{toc}{chapter}{Déroulement du projet}


\vspace*{3cm}

\begin{center}
	{\Huge\bfseries CHAPITRE \thechapter}
	
	\vspace{0.8cm}
	\rule{10cm}{1.2pt}
	
	\vspace{0.8cm}
	{\LARGE Déroulement du proje}
	
	\vspace{0.8cm}
	\rule{10cm}{1.2pt}
\end{center}

\vspace*{\fill}

\newpage

\section{Introduction}

Ce chapitre présente le déroulement du projet depuis sa planification initiale jusqu’à sa mise en œuvre finale. Il décrit l’organisation des tâches, la méthodologie adoptée et les différentes phases du développement. L’objectif est de montrer comment le projet a été structuré afin d’assurer une progression logique, maîtrisée et conforme aux objectifs pédagogiques du module « Applications Distribuées ».

\section{Planification opérationnelle}

Afin de garantir une réalisation efficace du projet, une planification opérationnelle a été mise en place dès le début du développement. Le travail a été découpé en tâches principales correspondant aux différentes fonctionnalités de l’application ainsi qu’aux aspects techniques liés à l’architecture distribuée.

Chaque tâche a été organisée selon son niveau de priorité, sa dépendance vis-à-vis des autres composants et sa complexité. Cette organisation a permis de suivre une progression cohérente tout en facilitant la validation intermédiaire des fonctionnalités développées.

\subsection{Tableau des tâches}

Le tableau suivant présente une vue synthétique des principales tâches du projet, leur description et leur objectif.

\vspace{0.5cm}

\textbf{(Espace réservé pour le tableau des tâches ou diagramme de Gantt — à compléter ultérieurement)}

\vspace{1cm}

\section{Processus de développement}

\subsection{Cycle de vie du développement}

Le développement de l’application a suivi une approche itérative et incrémentale. Chaque phase du projet a été réalisée progressivement, permettant de valider les fonctionnalités au fur et à mesure de leur implémentation.

Cette méthode favorise une meilleure compréhension des interactions entre les différentes couches de l’application, tout en réduisant les risques d’erreurs majeures lors de l’intégration finale.

Le choix de cette approche s’explique par :

\begin{itemize}
	\item la modularité de l’architecture de l’application,
	\item la possibilité de tester chaque composant indépendamment,
	\item la facilité d’intégration progressive des fonctionnalités.
\end{itemize}

\subsection{Phases du développement}

Le processus de développement a été structuré en plusieurs phases successives :

\textbf{Phase 1 : Analyse des besoins}

Cette phase a consisté à identifier les fonctionnalités essentielles d’une application e-commerce distribuée. Les besoins fonctionnels et non fonctionnels ont été définis afin d’établir une base claire pour le développement.

\textbf{Phase 2 : Conception de l’architecture}

Une architecture client-serveur a été conçue en séparant la couche présentation, la logique métier et la gestion des données. La sécurité de l’application a été intégrée dès cette étape.

\textbf{Phase 3 : Mise en place de l’environnement}

Les outils de développement ont été configurés, incluant le framework Spring Boot, la base de données MySQL et l’environnement de test. Cette étape a permis de préparer une base stable pour le codage.

\textbf{Phase 4 : Développement des fonctionnalités}

Les modules principaux ont été implémentés progressivement :

\begin{itemize}
	\item gestion des produits,
	\item gestion des utilisateurs,
	\item panier et commandes,
	\item authentification sécurisée.
\end{itemize}

Chaque fonctionnalité a été testée avant son intégration avec le reste du système.

\textbf{Phase 5 : Intégration de la sécurité}

Un mécanisme d’authentification basé sur JWT a été mis en place afin de contrôler l’accès aux différentes ressources et garantir la protection des données.

\textbf{Phase 6 : Tests et validation}

Des tests fonctionnels ont été réalisés pour vérifier le bon fonctionnement de l’application, la cohérence des données et la stabilité globale du système.

\textbf{Phase 7 : Documentation}

La dernière étape a consisté à rédiger le rapport du projet afin de documenter les choix techniques, la méthodologie adoptée et les résultats obtenus.

\section{Conclusion}

Ce chapitre a présenté l’organisation et la méthodologie suivies pour mener à bien le projet. La planification structurée et l’approche incrémentale ont permis de garantir une progression cohérente et une intégration efficace des fonctionnalités. Ces éléments constituent une base solide pour aborder les aspects de conception et de modélisation présentés dans le chapitre suivant.
